\usepackage{xcolor}
\usepackage{afterpage}
\usepackage{pifont,mdframed}
\usepackage[bottom]{footmisc}


\createsection{\Grader}{Grader di prova}

\newcommand{\inputfile}{\texttt{stdin}}
\newcommand{\outputfile}{\texttt{stdout}}

\newenvironment{warning}
  {\par\begin{mdframed}[linewidth=2pt,linecolor=gray]%
    \begin{list}{}{\leftmargin=1cm
                   \labelwidth=\leftmargin}\item[\Large\ding{43}]}
  {\end{list}\end{mdframed}\par}

% % % % % % % % % % % % % % % % % % % % % % % % % % % % % % % % % % % % % % % % % % %
% % % % % % % % % % % % % % % % % % % % % % % % % % % % % % % % % % % % % % % % % % %


	A Hogwarts, la più prestigiosa scuola di magia del mondo, si sa che \emph{alle scale piace cambiare}! Dopo un lungo viaggio in treno e la cerimonia di smistamento, sei pronto per la tua prima lezione: Pozioni. Il castello ha $N$ sale, numerate da $0$ a $N-1$, tra loro collegate da $M$ scale. Il tuo dormitorio si trova nella sala $0$ e la lezione di Pozioni si tiene nell'aula allestita nella stanza $N-1$, nei sotterranei del castello. Fortunatamente, prima di intraprendere il percorso attraverso le sale, conosci l'orario in cui compare e quello in cui scompare ogni scala. Per percorrere una scala impieghi esattamente 1 minuto, ma puoi sostare nelle sale per tutto il tempo che ritieni necessario. 
	
	Il professore di Pozioni è molto severo e non tollera ritardi, dunque devi assolutamente arrivare a lezione nel minor tempo possibile. Trova un modo per raggiungere l'aula nel minimo tempo possibile, se un modo per raggiungerla esiste!
	%se le scale non ti consentono di raggiungere l'aula, riceverai una punizione e verranno tolti punti alla tua casa.

% % % % % % % % % % % % % % % % % % % % % % % % % % % % % % % % % % % % % % % % % % %
% % % % % % % % % % % % % % % % % % % % % % % % % % % % % % % % % % % % % % % % % % %

\Implementation

Dovrai sottoporre un unico file, con estensione \texttt{.c}, \texttt{.cpp} o \texttt{.pas}.

\begin{warning}
Tra gli allegati a questo task troverai un template \texttt{hogwarts.c}, \texttt{hogwarts.cpp}, \texttt{hogwarts.pas} con un esempio di implementazione.
\end{warning}

Dovrai implementare la seguente funzione:

\begin{center}\begin{tabularx}{\textwidth}{|c|X|}
\hline
C/C++  & \verb|int raggiungi(int N, int M, int A[], int B[], int inizio[], int fine[]);|\\
\hline
Pascal  & \verb|function raggiungi(N,M: longint; A,B,inizio,fine: array of longint): longint;|\\
\hline
\end{tabularx}\end{center}

\begin{itemize}[nolistsep]
  \item L'intero $N$ rappresenta il numero di sale del castello.
  \item L'intero $M$ rappresenta il numero di scale.
  \item Gli array \texttt{A}, \texttt{B}, \texttt{inizio} e \texttt{fine} sono indicizzati da $0$ a $M-1$ e contengono le informazioni sulla comparsa e sparizione delle scale: l'$i$-esima scala collega tra loro le sale \texttt{A[$i$]} e \texttt{B[$i$]}, compare al tempo \texttt{inizio[$i$]} e scompare al tempo \texttt{fine[$i$]}.
  \item La funzione deve restituire il minimo tempo necessario per andare dalla sala $0$ alla sala $N-1$; se non è possibile raggiungere la sala $N-1$, deve restituire il valore $-1$.
\end{itemize}

\medskip

Il grader chiamerà la funzione \texttt{raggiungi} e ne stamperà il valore restituito sul file di output.

% % % % % % % % % % % % % % % % % % % % % % % % % % % % % % % % % % % % % % % % % % %
% % % % % % % % % % % % % % % % % % % % % % % % % % % % % % % % % % % % % % % % % % %


\Grader
Nella directory relativa a questo problema è presente una versione semplificata del grader usato durante la correzione, che potete usare per verificare le vostre soluzioni in locale. Il grader di esempio legge i dati da \inputfile{}, chiama la funzione che dovete implementare e scrive su \outputfile{}, secondo il seguente formato.

Il file di input è composto da $M+1$ righe, contenenti:
\begin{itemize}[nolistsep,itemsep=2mm]
\item Riga $1$: gli interi $N$ e $M$.
\item Righe $2, \ldots, M+1$: l'$i$-esima di queste righe contiene, nell'ordine, i valori \texttt{A[$i$]}, \texttt{B[$i$]}, \texttt{inizio[$i$]} e \texttt{fine[$i$]} per $i = 0, \ldots, M-1$.
\end{itemize}

Il file di output è composto da un'unica riga, contenente:
\begin{itemize}[nolistsep,itemsep=2mm]
\item Riga $1$: il valore restituito dalla funzione \texttt{raggiungi}.
\end{itemize}

% % % % % % % % % % % % % % % % % % % % % % % % % % % % % % % % % % % % % % % % % % %
% % % % % % % % % % % % % % % % % % % % % % % % % % % % % % % % % % % % % % % % % % %


\Constraints 

\begin{itemize}[nolistsep, itemsep=2mm]
	\item $2 \le N \le 500\,000$.
	\item $1 \le M \le 1\,000\,000$.
	\item $0 \le \texttt{A[}i\texttt{]}, \texttt{B[}i\texttt{]} \le N-1$ per ogni $i=0,\ldots, M-1$.
	\item Non ci sono scale che collegano una sala a se stessa ($\texttt{A[}i\texttt{]} \neq \texttt{B[}i\texttt{]}$).
	\item Non ci sono due o più scale che collegano le stesse due sale.
    \item Ogni scala è percorribile in una qualunque delle due direzioni.
	\item $0 \le \texttt{inizio[}i\texttt{]} < \texttt{fine[}i\texttt{]} \le 2\,000\,000$.
\end{itemize}

% % % % % % % % % % % % % % % % % % % % % % % % % % % % % % % % % % % % % % % % % % %
% % % % % % % % % % % % % % % % % % % % % % % % % % % % % % % % % % % % % % % % % % %


\Scoring

Il tuo programma verrà verificato su diversi test case raggruppati in subtask.
Per ottenere il punteggio relativo a un subtask, è necessario risolvere correttamente tutti i test che lo compongono.

\begin{itemize}[nolistsep,itemsep=2mm]
  \item \textbf{\makebox[2cm][l]{Subtask 1} [\phantom{1}0 punti]}: Casi d'esempio.
  \item \textbf{\makebox[2cm][l]{Subtask 2} [10 punti]}: $N \le 10, M \le 15$ e $\texttt{fine[}i\texttt{]} \le 20$ per ogni $i$.
  \item \textbf{\makebox[2cm][l]{Subtask 3} [21 punti]}: Tutte le scale sono fisse, cioè il tempo di inizio è $0$ per tutte e il tempo di fine è uguale per tutte.
  \item \textbf{\makebox[2cm][l]{Subtask 4} [18 punti]}: Le scale scompaiono soltanto, cioè il tempo di inizio è $0$ per tutte.
  \item \textbf{\makebox[2cm][l]{Subtask 5} [22 punti]}: $N \leq 1000$, $M \le 2000$, $\texttt{fine[}i\texttt{]} \le 5000$ per ogni $i$.
  \item \textbf{\makebox[2cm][l]{Subtask 6} [29 punti]}: Nessuna limitazione specifica.
\end{itemize}

% % % % % % % % % % % % % % % % % % % % % % % % % % % % % % % % % % % % % % % % % % %
% % % % % % % % % % % % % % % % % % % % % % % % % % % % % % % % % % % % % % % % % % %


\Examples

\begin{example}
\exmpfile{hogwarts.input0.txt}{hogwarts.output0.txt}%
\exmpfile{hogwarts.input1.txt}{hogwarts.output1.txt}%
\end{example}

% % % % % % % % % % % % % % % % % % % % % % % % % % % % % % % % % % % % % % % % % % %
% % % % % % % % % % % % % % % % % % % % % % % % % % % % % % % % % % % % % % % % % % %


\Explanation

Nel \textbf{primo caso di esempio} il modo più veloce per andare dalla sala $0$ alla $3$ è aspettare $1$ minuto, poi prendere la scala che collega $0$ e $1$ (ci metti $1$ minuto) e poi prendere immediatamente la scala che collega $1$ e $3$ (anche qui ci metti $1$ minuto), impiegando in totale $3$ minuti per arrivare a lezione.\\[2mm]

\begin{figure}[!h]
	\centering
	\includegraphics[width=0.245\textwidth]{asy_hogwarts/fig1a.pdf}
	\includegraphics[width=0.245\textwidth]{asy_hogwarts/fig1b.pdf}
	\includegraphics[width=0.245\textwidth]{asy_hogwarts/fig1c.pdf}
	\includegraphics[width=0.245\textwidth]{asy_hogwarts/fig1d.pdf}
\end{figure}

Nel \textbf{secondo caso di esempio} non è possibile andare dalla sala 0 alla sala 2! Infatti dovresti necessariamente passare per la sala $1$ perché non c'è mai una scala che collega direttamente $0$ e $2$. Al tempo $3$ compare una scala per andare da $0$ a $1$, dunque puoi trovarti nella stanza $1$ al più presto dopo $4$ minuti, e in quell'istante scompare la scala che collega le sale $1$ e $2$, impedendoti di raggiungere la destinazione.

\begin{figure}[!h]
	\centering
	\includegraphics[width=0.245\textwidth]{asy_hogwarts/fig2a.pdf}
	\includegraphics[width=0.245\textwidth]{asy_hogwarts/fig2b.pdf}
	\includegraphics[width=0.245\textwidth]{asy_hogwarts/fig2c.pdf}
	\includegraphics[width=0.245\textwidth]{asy_hogwarts/fig2d.pdf}
\end{figure}

\begin{solution}
	\newpage
\setcounter{figure}{0}

\lstset{
  numbers = left,
  numberstyle=\tiny\color{gray},
  backgroundcolor=\color{white},
  commentstyle=\color{red},
  basicstyle=\footnotesize\ttfamily,
  keywordstyle=\color{blue},
  xleftmargin=1cm,
  language=C
}

\definecolor{backcolor}{gray}{0.95}
\pagecolor{backcolor}
\createsection{\Solution}{Soluzione}
\createsection{\ArchiFissi}{{\small{$\blacksquare$}} \normalsize Se le scale sono fisse}
\createsection{\ScompaionoSolo}{{\small{$\blacksquare$}} \normalsize Se le scale possono soltanto scomparire}
\createsection{\DijkstraModificato}{{\small{$\blacksquare$}} \normalsize Soluzione $O(M\log M)$}
\createsection{\SoluzioneLineare}{{\small{$\blacksquare$}} \normalsize Soluzione $O(M+T_{\max})$}
\createsection{\CppSolution}{Esempio di codice \texttt{C++11}}

\newcommand{\inizio}{\ensuremath{\mathrm{inizio}}}
\newcommand{\fine}{\ensuremath{\mathrm{fine}}}

\Solution

Presentiamo due soluzioni complete del problema, ma prima vediamo le soluzioni di alcuni subtask che contengono idee utili per le soluzioni ottimali. Nel corso di questa nota, chiamiamo $T_{\max}$ il tempo di scomparsa dell'ultima scala. 

\ArchiFissi

Se tutte le scale sono fisse, cioè compaiono tutte al tempo $0$ e scompaiono tutte al tempo $T_{\max}$, il problema si risolve con una BFS (Breadth First Search o ricerca in ampiezza) sul grafo i cui vertici sono le sale e i cui archi sono le scale. Il \emph{peso} di un arco è il tempo impiegato per percorrere la scala corrispondente, dunque tutti gli archi hanno peso $1$. Quindi tramite una BFS è possibile sapere se è possibile andare dal vertice $0$ al vertice $N-1$ e, nel caso sia possibile, il minimo tempo $t$ necessario per farlo. L'unico accorgimento è che se $t > T_{\max}$ le scale scompaiono prima che si riesca ad arrivare alla stanza $N-1$ e quindi la risposta dovrà essere $-1$. Questa è la soluzione del subtask 3 e ha complessità computazionale lineare nel numero di archi.

\ScompaionoSolo

Se le scale sono tutte presenti all'inizio, ma possono scomparire, si può risolvere il problema modificando la BFS: quando ci troviamo in un nodo $x$ a distanza $d$ dal vertice $0$ e vogliamo percorrere un arco che va da $x$ a $y$, lo possiamo percorrere solo se $d$ è strettamente minore del momento in cui scompare l'arco in questione. Anche questa soluzione ha complessità computazionale lineare nel numero di archi.

\DijkstraModificato

Presentiamo ora una prima soluzione del problema con archi che possono sia comparire che scomparire. Il punto di partenza è l'algoritmo di Dijkstra per trovare il cammino di lunghezza minima su un grafo pesato con pesi non negativi. È sufficiente modificare l'algoritmo nel seguente modo: quando ci si ``espande'' da un nodo $x$ che sta a distanza $t$ dal vertice $0$ percorrendo un arco $(x, y)$ che compare al tempo $T_{\inizio}$ e scompare al tempo $T_{\fine}$, si controlla innanzitutto se $T_{\fine}$ è minore o uguale a $t$ (cioè se l'arco è già scomparso). In caso di risposta negativa, bisogna verificare se il collegamento con $y$ è già presente al tempo $t$ oppure no, dunque il tempo in cui si può raggiungere $y$ sarà $1$ più il massimo tra $T_{\inizio}$ e $t$. Questa soluzione ha complessità $O(M\log M)$.

%Per scegliere il nodo con distanza minore tra quelli non ancora visitati, si può usare una ricerca lineare (in questo modo la complessità computazionale totale è $O(N^2)$) oppure una coda di priorità (in questo modo la soluzione ha complessità computazionale $O(M \log M)$).

\SoluzioneLineare

Presentiamo ora la seconda soluzione del problema. L'idea è di trovare, per ogni tempo $t$, quali sono le sale che possono essere raggiunte \emph{esattamente} al tempo $t$ (e non prima). Al tempo $t = 0$, l'unica sala che può essere raggiunta è la sala $0$. Teniamo una variabile \texttt{vector <int> raggiunti[MAXT+1]} in cui in \texttt{raggiunti[$t$]} metteremo i nodi che sono raggiungibili esattamente al tempo $t$, più alcuni nodi raggiungibili prima di $t$ che vedremo come trattare in seguito.

Quando siamo al tempo $t$, vogliamo sapere quali nuovi vertici possiamo raggiungere a partire dai vertici in \texttt{raggiunti[$t$]} (che potranno essere raggiunti dal tempo $t+1$ in poi, a seconda di quando compaiono gli archi uscenti dai vertici di \texttt{raggiunti[$t$]}). Per farlo, consideriamo uno alla volta i vertici in \texttt{raggiunti[$t$]}. Sia $x$ un tale vertice. Per ogni arco $(x, y)$ che compare al tempo $T_{\inizio}$ e scompare al tempo $T_{\fine}$ possono verificarsi quattro situazioni differenti:
\begin{enumerate}
	\item la scala $(x, y)$ è inutile perché anche $y$ è raggiungibile entro il tempo $t$; in questo caso semplicemente ignoriamo l'arco;
	\item la scala $(x, y)$ non è più utilizzabile perché l'arco è scomparso prima che $x$ fosse raggiungibile, cioè $T_{\fine} \le t$; anche in questo caso ignoriamo l'arco;
	\item la scala $(x, y)$ è utile e percorribile immediatamente, allora dobbiamo segnare che $y$ è raggiungibile \emph{entro} il tempo $t+1$ (perché comunque ci mettiamo 1 secondo per percorrere la scala $(x, y)$, aggiungendo il nodo $y$ a \texttt{raggiunti[$t+1$]});
	\item la scala $(x, y)$ è utile ma \emph{al momento} non è utilizzabile immediatamente perché compare dopo il tempo $t$; in questo caso, detto $T_{\inizio}$ il momento in cui compare la scala, segneremo che $y$ è raggiungibile \emph{entro} il tempo $T_{\inizio}+1$ aggiungendo il nodo $y$ a \texttt{raggiunti[$T_{\inizio}+1$]}.
\end{enumerate}

Può accadere di segnare un nodo $x$ in più di un elemento dell'array di vector \texttt{raggiunti}; è quindi indispensabile tenere  una variabile \texttt{bool fatto[MAXN+1]} in cui nella casella $x$ viene messo \texttt{true} quando un nodo è raggiungibile per la prima volta (cioè la prima volta che, scorrendo i tempi $t$, incontriamo il nodo nella variabile \texttt{raggiunti[$t$]}). In questo modo, se incontriamo di nuovo il nodo $x$ sappiamo che dobbiamo ignorarlo.

Poiché guardiamo ogni arco al più una volta, e analizziamo ogni tempo $t$ da $0$ a $T_{\max}$, la complessità computazionale della soluzione proposta è $O(M + T_{\max})$.















\begin{comment}
Teniamo una coda che chiamiamo \texttt{nuovi\_nodi} in cui al tempo $t$ ci stanno i vertici raggiungibili esattamente al tempo $t$.
Vogliamo quindi trovare, nel modo più efficiente possibile, i vertici che sono raggiungibili \emph{esattamente} al tempo $t+1$. Questi possono provenire da due situazioni differenti:
\begin{enumerate}
	\item nodi $y$ tali che ci sia un arco $(x, y)$, con $x \in \texttt{nuovi\_nodi}$ e con l'arco $(x,y)$ comparso prima del tempo $t$ e che scomparirà dopo il tempo $t+1$;
	\item nodi $y$ tali che al tempo $t$ compare un arco $(x, y)$ dove $x$ è un qualsiasi vertice raggiungibile \emph{entro} il tempo $t$.
\end{enumerate}

La prima cosa da fare dunque è ordinare gli archi per tempo di comparsa, e questo si può fare sia utilizzando il \texttt{sort} delle stl del C++ sia tramite un \emph{counting sort}: la seconda soluzione è la preferibile in quanto la complessità computazionale è lineare in $T_{max}$. Per ogni nodo non ancora raggiungibile, ci interesserà memorizzare a quale sale è connesso e il tempo di scomparsa delle scale corrispondenti in una variabile \texttt{vector< pair<int, int> > Grafo[N]} (in cui il primo elemento della coppia indica la sala di arrivo e il secondo elemento il momento in cui scompare la scala).

Nella coda \texttt{nuovi\_nodi} al tempo $0$ inseriamo il nodo $0$. Per $t$ che va da $0$ a $T_{max}$:
\begin{itemize}
	\item Per ogni nodo presente in $\texttt{nuovi\_nodi[}t\texttt{]}$ guardo gli archi comparsi prima del tempo $t$ che escono da esso, che si trovano in \texttt{Grafo}. Sia $x$ il nuovo nodo, $y$ quello a cui è connesso dall'arco considerato. Possono verificarsi tre condizioni:
	\begin{enumerate}
		\item la scala $(x, y)$ è scomparsa strettamente prima del tempo $t+1$: la scala non è utilizzabile;
		\item la scala $(x, y)$ è inutile in quanto anche $y$ è raggiungibile entro il tempo $t$;
		\item la scala $(x, y)$ può essere percorsa: allora il nodo $y$ può essere raggiunto esattamente al tempo $t+1$.
	\end{enumerate}
	\item Per ogni arco che compare al tempo $t$, chiamiamo $x$ e $y$ le due sale che collega; possono verificarsi tre condizioni:
	\begin{enumerate}
		\item la scala $(x, y)$ è inutile perché sia $x$ sia $y$ sono raggiungibili entro il tempo $t$;
		\item la scala $(x, y)$ è utile e utilizzabile in quanto esattamente uno tra $x$ e $y$ è raggiungibile entro il tempo $t$, quindi l'altro tra $x$ e $y$ diventa raggiungibile per la prima volta esattamente al tempo $t+1$;
		\item la scala $(x, y)$ \emph{al momento} non è utilizzabile perché al tempo $t$ né $x$ né $y$ sono raggiungibili; in tal caso aggiungiamo la scala a \texttt{Grafo} in modo che se in futuro uno tra $x$ e $y$ diventerà raggiungibile, ci ricorderemo dell'esistenza di quella scala.
	\end{enumerate}
\end{itemize}

Durante tutto questo, se il nodo $N-1$ diventa raggiungibile possiamo direttamente uscire dalla funzione perché abbiamo la risposta al problema; invece se arriviamo al tempo $T_{max}$ senza aver raggiunto il nodo $N-1$ vuol dire che la funzione dovrà ritornare il valore $-1$. L'algoritmo che abbiamo presentato ha complessità computazionale $O(M + T_{max})$.
\end{comment}


\newpage
\CppSolution
Di seguito trovate un'implementazione della soluzione ottimale di questo problema.

%\lstinputlisting{./soluzione_hogwarts.cpp}
\colorbox{white}{\makebox[.99\textwidth][l]{
    \includegraphics[scale=.9]{soluzione_hogwarts.pdf}
}}

\afterpage{\nopagecolor}

\end{solution}
