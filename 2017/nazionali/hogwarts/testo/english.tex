\usepackage{xcolor}
\usepackage{afterpage}
\usepackage{pifont,mdframed}
\usepackage[bottom]{footmisc}


\createsection{\Grader}{Grader}

\newcommand{\inputfile}{\texttt{stdin}}
\newcommand{\outputfile}{\texttt{stdout}}

\newenvironment{warning}
  {\par\begin{mdframed}[linewidth=2pt,linecolor=gray]%
    \begin{list}{}{\leftmargin=1cm
                   \labelwidth=\leftmargin}\item[\Large\ding{43}]}
  {\end{list}\end{mdframed}\par}

% % % % % % % % % % % % % % % % % % % % % % % % % % % % % % % % % % % % % % % % % % %
% % % % % % % % % % % % % % % % % % % % % % % % % % % % % % % % % % % % % % % % % % %

	At Hogwarts, the most famous Wizardry School of the world, everyone knows that \emph{stairs like to change}! After a long train journey and the Sorting Ceremony, you are ready for your first class: \emph{Potions}. The castle has $N$ rooms, numbered from $0$ to $N-1$, which are connected with $M$ stairs. Your dormitory is located in room $0$ and the Potions class is in room $N-1$, at the basement of the castle. Luckily, before moving from your dormitory, you know the exact time when each staircase appears and disappears. It takes one minute for you to go up or down a staircase, but you can wait in the rooms for all the time you need.
	
	The teacher is very strict and does not tolerate delays, so you must get to the class in the least possible time. Calculate the minimum amount of time that you need to reach room $N-1$ from your dormitory, if a way to get there exists!

% % % % % % % % % % % % % % % % % % % % % % % % % % % % % % % % % % % % % % % % % % %
% % % % % % % % % % % % % % % % % % % % % % % % % % % % % % % % % % % % % % % % % % %

\Implementation

You shall submit one file having extension \texttt{.c}, \texttt{.cpp} or \texttt{.pas}.

\begin{warning}
Among the attachments of this task you will find a template (\texttt{hogwarts.c}, \texttt{hogwarts.cpp}, \texttt{hogwarts.pas}) with a sample incomplete implementation.
\end{warning}

You need to implement the following function:

\begin{center}\begin{tabularx}{\textwidth}{|c|X|}
\hline
C/C++  & \verb|int raggiungi(int N, int M, int A[], int B[], int inizio[], int fine[]);|\\
\hline
Pascal  & \verb|function raggiungi(N,M: longint; A,B,inizio,fine: array of longint): longint;|\\
\hline
\end{tabularx}\end{center}

\begin{itemize}[nolistsep]
  \item $N$ is an integer representing the number of rooms in the castle.
  \item $M$ is an integer representing the number of stairs.
  \item The arrays \texttt{A}, \texttt{B}, \texttt{inizio} and \texttt{fine} are indexed from $0$ to $M-1$ and they contain the information about the appearance and disappearance of the stairs: the $i$-th stair is between room \texttt{A[$i$]} and \texttt{B[$i$]}, appears at time \texttt{inizio[$i$]} and disappears at time \texttt{fine[$i$]}.
  \item The function has to return the minimum amount of time required to go from room $0$ to room $N-1$ if this is possible, otherwise it should return $-1$.
\end{itemize}

\medskip

The grader will call the function \texttt{raggiungi} and will print the returned value to the output file.

% % % % % % % % % % % % % % % % % % % % % % % % % % % % % % % % % % % % % % % % % % %
% % % % % % % % % % % % % % % % % % % % % % % % % % % % % % % % % % % % % % % % % % %


\Grader
In the directory for this problem there is a simplified version of the grader used during evaluation, which you can use to test your solutions locally. The sample grader reads data from \inputfile{}, calls the function that you should implement and writes to \outputfile{} in the following format.

The input file is made of $M+1$ lines, containing:
\begin{itemize}[nolistsep,itemsep=2mm]
\item Line $1$: integers $N$ and $M$.
\item Lines $2, \ldots, M+1$: the $i$-th line contains, in order, integers \texttt{A[$i$]}, \texttt{B[$i$]}, \texttt{inizio[$i$]} and \texttt{fine[$i$]} for $i = 0, \ldots, M-1$.
\end{itemize}

The output file is made of a single line, containing:
\begin{itemize}[nolistsep,itemsep=2mm]
\item Line $1$: the value returned by the function \texttt{raggiungi}.
\end{itemize}

% % % % % % % % % % % % % % % % % % % % % % % % % % % % % % % % % % % % % % % % % % %
% % % % % % % % % % % % % % % % % % % % % % % % % % % % % % % % % % % % % % % % % % %


\Constraints

\begin{itemize}[nolistsep, itemsep=2mm]
	\item $2 \le N \le 500\,000$.
	\item $1 \le M \le 1\,000\,000$.
	\item $0 \le \texttt{A[}i\texttt{]}, \texttt{B[}i\texttt{]} \le N-1$ for $i=0,\ldots, M-1$.
	\item There are no stairs that connect a room to itself ($\texttt{A[}i\texttt{]} \neq \texttt{B[}i\texttt{]}$).
	\item There cannot be any two stairs that connect the same two rooms.
	\item Each stair can be walked upward or downward.
	\item $0 \le \texttt{inizio[}i\texttt{]} < \texttt{fine[}i\texttt{]} \le 2\,000\,000$.
\end{itemize}

% % % % % % % % % % % % % % % % % % % % % % % % % % % % % % % % % % % % % % % % % % %
% % % % % % % % % % % % % % % % % % % % % % % % % % % % % % % % % % % % % % % % % % %


\Scoring 

Your program will be tested against several test cases grouped in subtasks.
In order to obtain the score of a subtask, your program needs to correctly solve all of its test cases.

\begin{itemize}[nolistsep,itemsep=2mm]
  \item \textbf{\makebox[2cm][l]{Subtask 1} [\phantom{1}0 punti]}: Examples.
  \item \textbf{\makebox[2cm][l]{Subtask 2} [10 punti]}: $N \le 10, M \le 15$ and $\texttt{fine[}i\texttt{]} \le 20$ for each $i$.
  \item \textbf{\makebox[2cm][l]{Subtask 3} [21 punti]}: All the stairs are fixed, i.e. the appearance time is $0$ for all of them and the disappearance time is the same for all.
  \item \textbf{\makebox[2cm][l]{Subtask 4} [18 punti]}: The stairs only disappear, i.e. the appearance time is $0$ for all of them.
  \item \textbf{\makebox[2cm][l]{Subtask 5} [22 punti]}: $N \leq 1000$, $M \le 2000$, $\texttt{fine[}i\texttt{]} \le 5000$ for each $i$. 
  \item \textbf{\makebox[2cm][l]{Subtask 6} [29 punti]}: No additional limitations.
\end{itemize}

% % % % % % % % % % % % % % % % % % % % % % % % % % % % % % % % % % % % % % % % % % %
% % % % % % % % % % % % % % % % % % % % % % % % % % % % % % % % % % % % % % % % % % %


\Examples

\begin{example}
\exmpfile{hogwarts.input0.txt}{hogwarts.output0.txt}%
\exmpfile{hogwarts.input1.txt}{hogwarts.output1.txt}%
\end{example}

% % % % % % % % % % % % % % % % % % % % % % % % % % % % % % % % % % % % % % % % % % %
% % % % % % % % % % % % % % % % % % % % % % % % % % % % % % % % % % % % % % % % % % %

\pagebreak

\Explanation

In the \textbf{first example} the fastest way to go from room $0$ to room $3$ is waiting for $1$ minute, taking the stair that connects $0$ to $1$ (it takes $1$ minute) and then taking immediately the stair that connects $1$ to $3$ (it takes $1$ minute as well), so you need $3$ minutes to get to your class.\\[2mm]

\begin{figure}[!h]
	\centering
	\includegraphics[width=0.245\textwidth]{asy_hogwarts/fig1a.pdf}
	\includegraphics[width=0.245\textwidth]{asy_hogwarts/fig1b.pdf}
	\includegraphics[width=0.245\textwidth]{asy_hogwarts/fig1c.pdf}
	\includegraphics[width=0.245\textwidth]{asy_hogwarts/fig1d.pdf}
\end{figure}

In the \textbf{second example} it is impossible to go from room $0$ to room $2$! In order to do that, you should pass through room $1$ because there is no stair connecting rooms $0$ and $2$ directly (at any time). At time $3$ a stair between rooms $0$ and $1$ appears, so you can be in room $1$ at least after $4$ minutes. However, in that moment the stair that connects rooms $1$ and $2$ disappears, so it is impossible for you to reach your destination.

\begin{figure}[!h]
	\centering
	\includegraphics[width=0.245\textwidth]{asy_hogwarts/fig2a.pdf}
	\includegraphics[width=0.245\textwidth]{asy_hogwarts/fig2b.pdf}
	\includegraphics[width=0.245\textwidth]{asy_hogwarts/fig2c.pdf}
	\includegraphics[width=0.245\textwidth]{asy_hogwarts/fig2d.pdf}
\end{figure}

\begin{solution}
	%\newpage
\setcounter{figure}{0}

\lstset{
  numbers = left,
  numberstyle=\tiny\color{gray},
  backgroundcolor=\color{white},
  commentstyle=\color{red},
  basicstyle=\footnotesize\ttfamily,
  keywordstyle=\color{blue},
  xleftmargin=1cm,
  language=C
}

\definecolor{backcolor}{gray}{0.95}
\pagecolor{backcolor}
\createsection{\Solution}{Solution}
\createsection{\ArchiFissi}{{\small{$\blacksquare$}} \normalsize If the stairs are fixed}
\createsection{\ScompaionoSolo}{{\small{$\blacksquare$}} \normalsize If the stairs can only disappear}
\createsection{\DijkstraModificato}{{\small{$\blacksquare$}} \normalsize Solution $O(M\log M)$}
\createsection{\SoluzioneLineare}{{\small{$\blacksquare$}} \normalsize Solution $O(M+T_{\max})$}
\createsection{\CppSolution}{Sample \texttt{C++11} code}

\newcommand{\inizio}{\ensuremath{\mathrm{start}}}
\newcommand{\fine}{\ensuremath{\mathrm{end}}}


\Solution

We present two optimal solutions for the problem, but first we will also show the solutions of some subtasks which contain useful ideas for the optimal solutions. From now on we will call $T_{\max}$ the last disappearance time for a stair. 


\ArchiFissi

If all the stairs are fixed, i.e. all of them appear at time $0$ and disappear at time $T_{\max}$, the problem can be solved with a BFS (Breadth First Search) on the graph whose nodes are rooms and whose edges are stairs. The \emph{weight} of an edge is the needed time to go through the corresponding stair, thus all the edges have weight $1$. Then, through a single BFS it is possible to know if we can go from node $0$ to $N-1$ and, in positive case, the minimum time $t$ which is necessary to do it. The only additional trick needed is that if $t > T_{\max}$ the stairs disappear before reaching room $N-1$ thus the answer should be $-1$. This algorithm solves subtask 3 and has linear complexity $O(M)$.


\ScompaionoSolo

If the stairs are all present at the beginning, but can disappear at different times, we can still solve the problem by modifying the BFS. When we consider a node $x$ at distance $d$ from node $0$ and an edge connecting $x$ to $y$, we are allowed to go through it only if $d$ is strictly less than the disappearing time of the interested edge. This solution also has linear complexity $O(M)$.


\DijkstraModificato

We now present the first solution of the problem with edges that can both appear and disappear, which is based on Dijkstra's algorithm for finding shortest paths on a graph with non-negatives weights. We change Dijkstra's algorithm in this way: when we ``expand'' a node $x$ at distance $t$ from node $0$ and consider an edge $(x, y)$ which appears at time $T_{\inizio}$ and disappears at time $T_{\fine}$, first of all we need to check if $T_{\fine}$ is less or equal to $d$ (i.e. if the edge has already disappeared). In case of negative answer, we also need to verify if the edge is already present at time $t$ or not, so that the time in which $y$ can be reached will be $1$ plus the maximum between $T_{\inizio}$ and $t$. This solution has complexity $O(M\log M)$.


\SoluzioneLineare

We now present the second solution to the problem. The idea is to find, for each time $t$, which rooms can be reached \emph{exactly} at time $t$ (and not before). At time $t = 0$, the only room that can be reached is room $0$. We then keep a variable \texttt{vector<int> reached[MAXT+1]} where \texttt{reached[$t$]} contains all the nodes that can be reached exactly at time $t$, plus other nodes that can be reached before $t$ which we will see how to deal with later.

When considering time $t$, we want to know which new nodes we can reach starting from nodes in \texttt{reached[$t$]} (that can be reached from time $t+1$ on, depending on when the corresponding stairs appear). To do so, we consider one by one the nodes in \texttt{reached[$t$]}. Let $x$ be such a node, then for each edge $(x, y)$ that appears at time $T_{\inizio}$ and disappears at time $T_{\fine}$ we can have four different situations:
\begin{enumerate}
	\item The stair $(x, y)$ is useless because $y$ can also be reached within time $t$. In this case we simply ignore the edge.
	\item The stair $(x, y)$ is useless because the edge disappeared before $x$ was reached, that is $T_{\fine} \le t$. Also in this case we ignore the edge.
	\item The stair $(x, y)$ is immediately useful, thus we need to record that $y$ can be reached in time $t+1$ (we still need 1 minute to go through the stair). In this case we add node $y$ to \texttt{reached[$t+1$]}.
	\item The stair $(x, y)$ is useful but \emph{not immediately}, since it appears after time $t$. In this case we record that $y$ can be reached \emph{within} time $T_{\inizio}+1$ adding node $y$ to \texttt{reached[$T_{\inizio}+1$]}.
\end{enumerate}

It is possible to add a node $x$ more than once in \texttt{reached}. It is therefore necessary to keep a variable \texttt{bool visited[MAXN+1]} where in index $x$ we put \texttt{true} if the vertex $x$ has been already reached for the first time (i.e. the first time that, scrolling the times $t$, we encounter the vertex in \texttt{reached[$t$]}). In this way, if we meet again node $x$ we know that we can ignore it.

Since we scan each edge at most once and analyze each time $t$ from $0$ to $T_{\max}$, the computational complexity of the proposed solution is $O(M + T_{\max})$.


\newpage
\CppSolution

%\lstinputlisting{./soluzione_hogwarts.cpp}
\colorbox{white}{\makebox[.99\textwidth][l]{
    \includegraphics[scale=.9]{soluzione_hogwarts_en.pdf}
}}

\afterpage{\nopagecolor}

\end{solution}
